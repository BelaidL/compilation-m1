%; whizzy paragraph
\documentclass[a4paper,8pt]{article}
\usepackage[francais]{babel}
\usepackage{fancybox}
%% Encodage et police
\usepackage{hyperref}
\usepackage{mathpartir}
\usepackage{ucs}
\usepackage[utf8x]{inputenc}
\usepackage{lmodern}
\usepackage[T1]{fontenc}
\usepackage{geometry}
\usepackage{xcolor}
\geometry{hmargin=0.7cm, vmargin=2.5cm}
%% BNF
\usepackage{tabularx}
\usepackage{array}
\def\colsep@widearray{\,}
\newcolumntype{C}{@{\colsep@widearray}>{\({}}c<{{}\)}@{\colsep@widearray}}
\newcolumntype{S}{@{\colsep@widearray}>{{}}r<{{}}@{\colsep@widearray}}
\newcolumntype{L}{@{\colsep@widearray}>{{}}l<{{}}@{\colsep@widearray}}
\newcolumntype{R}{@{\colsep@widearray}>{\sl{}}r<{{}}@{\colsep@widearray}}
\newcolumntype{Z}{@{\colsep@widearray}>{{}}X<{{}}@{\colsep@widearray}}
\newcommand{\comment}[1]{\hspace{3cm}\hfill \mbox{\textit{#1}}}
\newenvironment{BNF}[1][\linewidth]%
{\begin{math}\begin{array}{rclr}}%
{\end{array}\end{math}}%
\newcommand{\kwd}[1]{\texttt{#1}}
\newcommand{\lex}[1]{\textsf{#1}}
\newcommand{\rul}[1]{#1}
\newcommand{\car}[1]{\texttt{#1}}
\newcommand{\plus}{$^+$ }
\newcommand{\mult}{$^*$ }
\newcommand{\opt}{$^*$ }
\newcommand{\repeatseq}[1]{\textsl{\meta{\{}} #1 \textsl{\meta{\}}}}
\newcommand{\repeatseqplus}[1]{\textsl{\meta{\{}} #1 \textsl{\meta{\}}}⁺}
\newcommand{\optw}[1]{\textsl{\meta{[}} #1 \textsl{\meta{]}}}
\newcommand{\optseq}[1]{\optw{#1}}

\newcommand\mvalue{v}
\newcommand\env{E}
\newcommand\store{M}
\newcommand\type{\tau}
\newcommand\otype{\tau^?}
\newcommand\meta[1]{\textrm{#1}}
\newcommand\expr{\meta{e}}
\newcommand\definition{\meta{d}}
\newcommand\vdefinition{\meta{d}_v}
\newcommand\tdefinition{\meta{d}_t}
\newcommand\program{\meta{p}}
\newcommand\typecon{\meta{T}}
\newcommand\tvar{\alpha}
\newcommand\rlabel{\ell}
\newcommand\id{x}
\newcommand\cid{K}
\newcommand\lint{n}
\newcommand\lchar{c}
\newcommand\lbool{b}
\newcommand\lstring{s}
\newcommand\pattern{\meta{m}}
\newcommand\branch{\meta{b}}
\newcommand\addr{a}
\newcommand\eval[5]{#1, #2 \vdash #3 \Downarrow #4, #5}
\newcommand\evalvdef[5]{#1, #2 \vdash #3 \Rightarrow #4, #5}
\newcommand\evalmatch[5]{#1, #2 \vdash #3 \sim #4 \Uparrow #5}
\newcommand\evalnotmatch[4]{#1, #2 \vdash #3 \not\sim #4}
\newcommand\caseanalysis[6]{#1, #2 \vdash #3 \sim #4 \Downarrow #5, #6}
\newcommand\unit{()}

\newlength\codewidth
\setlength\codewidth{15cm}
\newenvironment{code}[1][\codewidth]{
\begin{center}
\Sbox
\hspace{0.3cm}\minipage{#1}\small
}{
\endminipage
\endSbox\fbox{\TheSbox}
\end{center}
}

\newcommand\many[1]{\overline{#1}}

%%
\title{
\vspace{-1.5cm}
Projet du cours «~Compilation~» \\
Jalon 2~: Interprétation de \textsc{Hopix}
}
\date{\scriptsize version numéro \input{version}}

\begin{document}

\maketitle

\section{Syntaxe abstraite}

Dans les sections suivantes, lorsque l'on utilisera de la syntaxe
concrète dans les spécifications, on fera référence implicitement
aux arbres de syntaxe abstraites sous-jacents définis par la grammaire
suivante:

\begin{code}[17cm]
\begin{BNF}
\program  & $::=$  & \many{\definition} \comment{Programme} 
\\
\definition
& $::=$
& {\kwd{type}}\;
 \typecon \,\many\tvar := \tdefinition
 \comment{Définition de type}
\\
& | &
\kwd{extern}\; \id \,\car{:}\, \type
\comment{Déclaration d'une valeur externe}
\\
& | &  \vdefinition
\comment{Définition de valeur(s)}
\\
\tdefinition & $::=$ &
\Sigma
\many{\cid \car{:} \many\type}
\comment{Type somme}
\\
& | &
\bullet
\comment{Type abstrait}
\\
\vdefinition & $::=$ &
\kwd{val} \; \id \;\many{\pattern}\, \car{:} \otype  \;\car{=}\; \rul{\expr}
\comment{Valeur simple}
\\
& | &
\kwd{rec}\;
\many{\id\; \many{\pattern}\, \car{:} \otype  \car{=} \; \rul{\expr}}
\comment{Valeurs mutuellement récursives}
\\
\type
& $::=$ & \typecon \car{[} \many\type\car{]}
\comment{Type construit}\\
& | & \tvar
\comment{Variable de type}
\\
\expr      & $::=$ & \lint \comment{Entier} \\
          &   | & \lbool \comment{Booléen} \\
          &   | & \lchar \comment{Caractère} \\
          &   | & \lstring \comment{Chaîne de caractères} \\
          &   | & \id                            \comment{Variable} \\
          &   | & \cid \car{[} \many\type \car{]} \car{(} \many{\expr} \car{)}
                  \comment{Construction d'une donnée étiquetée} \\
          &   | & \car{(} \rul{\expr} \;\car{:}\; \type \car{)} \comment{Annotation de type} \\
          &   | & \rul{\vdefinition} \,\car{;}\; \rul{\expr}
                  \comment{Définition locale} \\
          &   | & \rul{\expr} \, \car{[} \many\type \car{]} \car{(} \many{\rul{\expr}} \car{)} \comment{Application} \\
          &   | & \car{$\backslash$} \car{[} \many\tvar \car{]} \car{(} \many{\rul{\pattern}} \car{)} \,\car{=>}\, \rul{\expr}
                  \comment{Fonction anonyme} \\
          &   | & \rul{\expr} \; \car{?} \; \many{\branch}  \comment{Analyse de motifs} \\
          &   | & \kwd{if} \; \rul{\expr} \; \kwd{then} \; \rul{\expr} \; \kwd{else} \; \rul{\expr} \comment{Conditionnelle} \\
          &   | & \kwd{ref} \; \rul{\expr} \comment{Création d'une référence} \\
          &   | & \rul{\expr} \; \car{:=} \; \rul{\expr} \comment{Modification d'une référence} \\
          &   | & ! \rul{\expr} \comment{Lecture d'une référence} \\
          &   | & \kwd{while}\, \car{(} \rul{\expr} \car{)} \; \rul{\expr} \comment{Boucle} \\
\\
\branch    & $::=$ & \rul{\pattern} \;\car{=>}\; \rul{\expr} \comment{Cas d'analyse} \\
\\
\pattern & $::=$ & \id \comment{Motif universel liant}\\
          &   | & \car{\_} \comment{Motif universel non liant}\\
          &   | & \car{(} \rul{\pattern} \;\car{:}\; \type \car{)} \comment{Annotation de type} \\
          &   | & \lint \comment{Entier} \\
          &   | & \lbool \comment{Booléen} \\
          &   | & \lchar \comment{Caractère} \\
          &   | & \lstring \comment{Chaîne de caractères} \\
          &   | & \cid \car{(} \many{\rul{\pattern}} \car{)}
                  \comment{Valeurs étiquetées}\\
          &   | & \rul{\pattern} \;\car{|}\; \rul{\pattern} \comment{Disjonction} \\
          &   | & \rul{\pattern} \;\car{\&}\; \rul{\pattern} \comment{Conjonction} \\
\end{BNF}\smallskip
\end{code}

Dans cette grammaire, on a utilisé les \textit{métavariables} suivantes:
\begin{itemize}

\item $\typecon$ pour représenter un constructeur de type.

\item $\id$ pour représenter un identificateur de valeur.

\item $\cid$ pour représenter le nom d'un constructeur de données.

\item $\type$ pour représenter un type.

\item $\expr$ pour représenter une expression.

\item $\pattern$ pour représenter un motif.

\item $\lint$ pour représenter un litéral de type entier.

\item $\lbool$ pour représenter un litéral de type booléen.
  
\item $\lchar$ pour représenter un litéral de type caractère.

\item $\lstring$ pour représenter un litéral de type chaîne de caractères.

\item $\tvar$ pour représenter une variable de type.

\item $\branch$ pour représenter une branche d'analyse de motifs.
  
\end{itemize}

Par ailleurs, on a aussi écrit $\many{X}$ pour représenter une liste potentiellement vide de $X$.

\section{Interprétation}

\subsection{Valeurs}

Les valeurs du langage sont définies par la grammaire suivante:

\begin{code}[17cm]
\begin{BNF}
  \mvalue & ::= &
\lint
\comment{Entier}
\\
& | &
\lbool
\comment{Booléen}
\\
& | &
\lchar
\comment{Caractère}
\\
& | &
\lstring
\comment{Chaîne de caractères}
\\
& | &
\unit
\comment{Unité}
\\
& | &
\cid (\many\mvalue)
\comment{Valeur étiquetée}
\\
& | &
\addr
\comment{Adresse d'une référence}
\\
& | &
(\many\pattern \;\car{=>}\; \expr) [\env]
\comment{Fermeture}
\\
& | &
\textrm{prim}
\comment{Primitive}
\end{BNF}
\end{code}

\paragraph{Environnement d'évaluation}
Les identificateurs de programme sont associés à des valeurs à l'aide
d'un environnement d'évaluation $\env$. On écrit «~$\env[\id]$~» pour
parler de la valeur de $\id$ dans l'environnement $\env$.  On écrit
«~$\env + \id \mapsto \mvalue$~» pour parler de l'environnement $\env$
étendu par l'association entre l'identificateur $\id$ et la valeur
$\mvalue$.

\paragraph{Références}
Les adresses des références sont allouées dynamiquement dans une
mémoire~$\store$.  On écrit «~$\store + \addr \mapsto \mvalue$~» pour
représenter la mémoire qui étend la mémoire $\store$ avec une nouvelle
adresse $\addr$ où se trouve stockée la valeur~$\mvalue$. On écrit
«~$\store[\addr \leftarrow \mvalue]$~» pour représenter la mémoire
$\store$ modifiée seulement à l'adresse $\addr$ en y stockant la
valeur $\mvalue$. Enfin, on écrit
«~$(\addr \mapsto \mvalue) \in \store$~» pour indiquer que la valeur
$\mvalue$ est stockée à l'adresse $\addr$ de la mémoire $\store$.

\paragraph{Primitive}
Les primitives du langage sont définies dans le module
\verb!HopixInterpreter!. Ce sont des fonctions \textsc{OCaml}
que l'on a rendues accessibles depuis \textsc{Hopix} ou bien
des valeurs primitives comme \textbf{true} ou \textbf{false}.

\subsection{Évaluation des programmes}

Un programme $\program$ s'évalue à partir d'une mémoire vide et d'un
environnement contenant les primitives du langage en évaluant
successivement les définitions de valeurs dans leur ordre d'apparition
dans le programme.

Pour spécifier précisément ce processus, il faut donc décrire la façon
dont les définitions de valeurs s'évaluent: c'est le rôle de la
section~\ref{sec:evaldef}. Les expressions sont les termes qui
s'évaluent en des valeurs. Leur évaluation est spécifiée dans la
section~\ref{sec:exp}.  Elle s'appuie sur l'évaluation de l'analyse
par cas (section~\ref{sec:case}) et l'analyse de motifs
(section~\ref{sec:pattern}).

\subsection{Évaluation des définitions}
\label{sec:evaldef}

L'évaluation des définitions s'appuie sur le jugement:
\[
  \evalvdef\env\store\vdefinition{\env'}{\store'} \\
\]
qui se lit «~Dans l'environnement $\env$ et la mémoire $\store$, la
définition $\vdefinition$ étend $\env$ et $\env'$ et modifie $\store$
en $\store'$.

\textbf{Cette partie de la spécification est omise volontairement.}
Vous devez réfléchir à une spécification raisonnable. 


\subsection{Évaluation des expressions}
\label{sec:exp}

Le jugement d'évaluation des expressions s'écrit:
\[
\eval\env\store\expr\mvalue{\store'}
\]
et se lit
«~Dans l'environnement d'évaluation $\env$,
  l'expression $\expr$ s'évalue en $\mvalue$ et change la mémoire $\store$ en $\store'$.~»

  Dans la suite, les termes sont allégés de leurs types car ces derniers n'influent pas sur
  l'évaluation. Le jugement d'évaluation est défini par les règles suivantes:
\begin{mathpar}
  \infer{
  }{
    \eval\env\store\lint\lint\store
  }

  \infer{
  }{
    \eval\env\store\lchar\lchar\store
  }

  \infer{
  }{
    \eval\env\store\lbool\lbool\store
  }

  \infer{
  }{
    \eval\env\store\lstring\lstring\store
  }

  \infer{
    \env(\id) = \mvalue
  }{
    \eval\env\store\id\mvalue\store
  }

  \infer{
    \store_0 = \store
    \\
    \forall i \in [1 \ldots n], \eval\env{\store_{i-1}}{\expr_i}{\mvalue_i}{\store_{i}}
  }{
    \eval\env\store{\cid (\expr_1, \ldots, \expr_n)}{\cid (\mvalue_1, \ldots, \mvalue_n)}{\store_{n}}
  }

  \infer{
    \evalvdef\env\store\vdefinition{\env'}{\store'} \\
    \eval{\env'}{\store'}\expr\mvalue{\store''}
  }{
    \eval\env\store{\vdefinition; \expr}\mvalue{\store''}
  }

  \infer{
    \eval\env\store{\expr_f}{(\pattern_1 \cdots \pattern_n \car{ => } \expr)[\env']}{\store'}
    \\
    \store_0 = \store'
    \\
    \forall i \in [1 \ldots n], \eval\env{\store_{i-1}}{\expr_i}{\mvalue_i}{\store_i}
    \\
    \env_0 = \env'
    \\
    \forall i \in [1 \ldots n], \evalmatch{\env_{i-1}}{\store_n}{\pattern_i}{\mvalue_i}{\env_i}
    \\
    \eval{\env_n}{\store_n}{\expr}\mvalue{\store'}
  }{
    \eval\env{\store}{\expr_f\, (\expr_1, \cdots, \expr_n)}\mvalue{\store'}
  }

  \infer{
    \eval\env\store\expr\addr{\store'}
    \\
    (\addr \mapsto \mvalue) \in \store'
  }{
    \eval\env\store{!\expr}\mvalue{\store'}
  }

  \infer{
    \eval\env\store\expr\addr{\store'}
    \\
    \eval\env{\store'}{\expr'}{\mvalue}{\store''}
  }{
    \eval\env\store{\expr \car{ := } \expr'}\unit{\store''[\addr \mapsto \mvalue]]}
  }

  \infer{
    \eval\env\store{\expr}{\mvalue_s}{\store'}
    \\
    \caseanalysis\env{\store'}{\mvalue_s}{\many{\branch}}{\mvalue}{\store''}
  }{
    \eval\env\store{\expr \car{ ? } \many{\branch}}{\mvalue}{\store''}
  }

  \infer{
    \eval\env\store{\expr_b}{\textbf{true}}{\store'} \\
    \eval\env{\store'}{\expr}{\unit}{\store''} \\
    \eval\env{\store''}{\kwd{while }(\expr_b)\; \expr}\unit{\store'''}
  }{
    \eval\env\store{\kwd{while }(\expr_b)\; \expr}\unit{\store'''}
  }

  \infer{
    \eval\env\store{\expr_b}{\textbf{false}}{\store'}
  }{
    \eval\env\store{\kwd{while }(\expr_b)\; \expr}\unit{\store'}
  }

\end{mathpar}

\subsection{Analyse par cas}
\label{sec:case}

L'analyse par cas d'une valeur $\mvalue$ consiste à traiter une liste
de cas représentés par des branches $\many\branch$ de la forme
«~$\pattern \car{ => } \expr$~» en évaluant la première de ces branches
dont le motif capture la valeur $\mvalue$. Ce dernier mécanisme
est représenté dans la section suivante.

\begin{mathpar}
  \infer{
    \evalmatch\env\store\mvalue\pattern{\env'}
    \\
    \eval{\env'}\store\expr{\mvalue'}{\store'}
  }{
    \caseanalysis\env\store\mvalue{(\pattern\car { => } \expr) \many\branch}{\mvalue'}{\store'}
  }

  \infer{
    \evalnotmatch\env\store\mvalue\pattern
    \\
    \caseanalysis\env\store\mvalue{\many\branch}{\mvalue'}{\store'}
  }{
    \caseanalysis\env\store\mvalue{(\pattern\car { => } \expr) \many\branch}{\mvalue'}{\store'}
  }
\end{mathpar}

\subsection{Analyse de motifs}
\label{sec:pattern}

L'évaluation des motifs s'appuie sur le jugement:
\[
   \evalmatch\env\store\mvalue\pattern{\env'}
\]
qui se lit «~Dans l'environnement $\env$
et dans la mémoire $\store$, la valeur $\mvalue$ est capturée
par le motif~$\pattern$ en étendant l'environnement $\env$ en
$\env'$.~»

\textbf{Cette partie de la spécification est omise volontairement.}
Vous devez réfléchir à une spécification raisonnable. 

\section{Travail à effectuer}

La seconde partie du projet est l'écriture de l'interprète de
la sémantique opérationnelle décrite plus haut.

La compilation s'effectue par la commande«~\verb!make!~».

Le projet est à rendre \textbf{avant le}~:

\begin{code}
\begin{center}
\large\textbf{21 janvier 2016 à 23h59}
\end{center}
\end{code}

Pour finir, vous devez vous assurer des points suivants~:
\begin{code}
\begin{itemize}
\item Le projet contenu dans cette archive \textbf{doit compiler}.
\item Vous devez \textbf{être les auteurs} de ce projet.
\item Il doit être rendu \textbf{à temps}.
\end{itemize}
\end{code}

Si l'un de ces points n'est pas respecté, la note de $0$ vous sera affectée.

\section{Log}

\input{changes}

\end{document}
